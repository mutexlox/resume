% LaTeX file for resume
% This file uses the resume document class (res.cls)

\documentclass[margin]{res} \usepackage{hyperref}
\usepackage[none]{hyphenat}
% the margin option causes section titles to appear to the left of body text
%\textwidth=5.7in % increase textwidth to get smaller right margin
%\usepackage{helvetica} % uses helvetica postscript font (get helvetica.sty)
%\usepackage{newcent}   % uses new century schoolbook postscript font

\newcommand{\course}[4]{ %
    %\begin{minipage}{0.6\resumewidth}
    \textbf{#1}: \textit{#2} (#4)\\
      #3
      \vspace{1em}
  %\end{minipage} & \begin{minipage}{1em}
  %    #3\\
  %    \vphantom{#2}
  %    \vspace{1em}
 % \end{minipage} \\}
    \\}
\begin{document}

\name{Miriam Zimmerman}

\address{
  \href{mailto:miriam@mutexlox.com}{miriam@mutexlox.com} \\
}
\address{  % The \hfills are to right-align the lines
  \hfill \href{https://mutexlox.com}{https://mutexlox.com} \\
}

\begin{resume}

% \section{Interests} Kernel and other low-level programming and security (primarily binary exploitation, reverse engineering, and cryptography)

\section{Experience}
  \textbf{Google: Senior Software Engineer}, ChromeOS \hfill 2022 --
  \begin{minipage}{0.84\textwidth}
    \vspace{0.2em}
    \begin{itemize} \itemsep -1pt
        \item Tech Lead for ChromeOS Experimentation
        \item Led implementation for early-boot A/B experimentation:
          \vspace{-0.5em}
          \begin{itemize} \itemsep -1pt
              \item Collaborated with Product Manager on outreach to
                prospective client teams, ensuring our solution met their needs
              \item Mentored new hire junior engineer, delegating work to them
              \item Implemented complex, safety-critical pieces, e.g.
                disaster recovery
          \end{itemize}
        \item Performed reviews of configuration changes for A/B testing,
          checking for safety, proper data analysis, and code health
    \end{itemize}
  \end{minipage}

  \textbf{Google: Software Engineer}, ChromeOS \hfill 2018 -- 2022
  \begin{minipage}{0.84\textwidth}
    \vspace{0.2em}
    \begin{itemize} \itemsep -1pt
        \item Designed, planned, and assigned work for long-term,
          multi-engineer early-boot A/B experimentation project, prioritizing
          safety and revovery from buggy experiments
       \item Designed and implemented late-boot A/B experimentation outside of
          browser, working with client teams to ensure their needs were met
        \item Designed and implemented system to upload crashes that occur
          during integration testing, collaborating with test lab infrastructure
          teams
        \item Implemented integration tests for crash reporting, detecting many
          bugs
        \item Implemented alerts to notify teams when code has high-rate crashes
        \item Improved reliability of crash reporting system by implementing
          metrics, fixing bugs, and managing server load
        \item Ported interrupt-handling devices from kernel to userspace
          (in Rust)
    \end{itemize}
  \end{minipage}

  \textbf{Google: Software Engineer}, Cloud Security \hfill 2015 -- 2018
  \begin{minipage}{0.84\textwidth}
    \vspace{0.2em}
    \begin{itemize} \itemsep -1pt
        \item Worked on Trusted Platform Module code and related software:
          \vspace{-0.5em}
          \begin{itemize} \itemsep -1pt
            \item Found bugs in Linux TPM driver; wrote and upstreamed fixes
            \item Wrote code to generate Endorsement Key certificates
            \item Implemented TPM register interface to public specification
            \item Dramatically increased test coverage and added automated fuzzing
          \end{itemize}
        \item Substantially increased difficulty for an attacker that breaks
          out of a virtual machine to move laterally to other production
          machines
        \item Evaluated several options for detecting Spectre and Meltdown:
          \vspace{-0.5em}
          \begin{itemize} \itemsep -1pt
            \item Reviewed research on side-channel cache attack detection
            \item Proposed detection methods using performance counters
            \item Measured false positive/negative rates for proposed schemes
            \item Wrote a widely-read internal report on detection options
          \end{itemize}
        \item Designed, implemented, and evaluated several hardware
          performance counter options to detect specific
          malicious behavior on GCE
        \item Automated test process for a minimal Type-I hypervisor and
          reduced time to run tests from roughly 30 minutes to roughly 5
        \item Developed a plan to improve inclusiveness in the Cloud Security
          organization as it grows, shared it with the directors,
          and met with one director to discuss in depth and further develop it
    \end{itemize}
  \end{minipage}

  \textbf{TEALS: Volunteer Teacher} \hfill 2017 -- 2021
  \begin{minipage}{0.84\textwidth}
    \vspace{0.2em}
    \begin{itemize} \itemsep -1pt
      \item Crafted and delivered lectures on Advanced Placement CS topics
        to high school students in an underprivileged school
    \end{itemize}
  \end{minipage}

\section{Education}
  \textbf{Carnegie Mellon University}, School of Computer Science
    \hfill May 2015\\
  B.S. in Computer Science with University Honors (GPA: 3.82)

\section{Languages}
  C++, C, Python, Go, bash, x86 assembly, Rust, Java, SML, \LaTeX, JavaScript

\section{Tools}
  vim, gdb, bazel, googletest, Abseil C++ libs, protobuf, git, Mercurial, svn, Eclipse

\end{resume}
\end{document}

